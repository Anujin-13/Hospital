\documentclass[12pt]{article}
\usepackage[utf8]{inputenc}
\usepackage[mongolian]{babel}
\usepackage{graphicx}
\usepackage{geometry}
\usepackage{hyperref}
\geometry{a4paper, margin=3cm}

\begin{document}

\begin{center}
\textbf{ШИНЖЛЭХ УХААН ТЕХНОЛОГИЙН ИХ СУРГУУЛЬ} \\[0.5cm]
\textbf{Мэдээлэл, Холбооны Технологийн Сургууль} \\[4cm]

\textbf{\Large ТАЙЛАН} \\[0.5cm]
\textbf{Програмчлалын үндэс дадлага} \\[0.5cm]
2025–2026 оны хичээлийн жилийн хавар \\[2cm]
\end{center}

\vspace{1cm}

\noindent\textbf{Тайлан ажлын нэр:} \hspace{1cm} Эмнэлгийн өвчтөн бүртгэлийн систем \\[0.5cm]
\textbf{Хичээл заасан багш:} \hspace{1cm} А.Отгонбаяр \\[0.5cm]
\textbf{Бие даалтын ажил гүйцэтгэсэн:} \\
\hspace{1cm} Э.Анужин B242270109 \\
\hspace{1cm} Ө.Анар B242270129 \\

\vfill

\begin{center}
Улаанбаатар хот, 2025 он.
\end{center}

\newpage

\section*{Төслийн танилцуулга}

Энэхүү \textbf{“Эмнэлгийн Өвчтөн Удирдлагын Систем”} нь эмнэлгийн дотоод үйл ажиллагааг цахимжуулж, өвчтөн бүртгэл, цаг захиалга болон эмчилгээний түүхийг удирдах зорилготой. Систем нь хэрэглэгчдэд хурдан, зохион байгуулалттай үйлчилгээ үзүүлэх боломжийг бүрдүүлдэг бөгөөд \textbf{Java} хэл дээр бүтээгдсэн.

\subsection*{Хэрэглэгчид ба үүрэг}

\begin{itemize}
  \item \textbf{Өвчтөн} — Өөрийн мэдээллийг бүртгүүлэх, эмчид цаг захиалах, эмчилгээний түүхээ харах.
  \item \textbf{Эмч} — Өвчтөний мэдээллийг харах, онош бичих, түүх хөтлөх.
  \item \textbf{Админ / Эмнэлгийн ажилтан} — Бүх бүртгэлийг хянах, цагийн хуваарь удирдах, тайлан гаргах.
\end{itemize}

\subsection*{Гол функцүүд}

\begin{itemize}
  \item Өвчтөн нэмэх, засах, устгах
  \item Эмчийн мэдээлэл бүртгэх
  \item \texttt{Appointment} ашиглан цаг захиалга хийх
  \item \texttt{VisitRecord} ашиглан онош, эмчилгээний түүх хадгалах
  \item Өвчтөн болон эмчийн мэдээлэл хайх, жагсааж харах
\end{itemize}

\subsection*{Кодын дизайн, бүтэц}

Системийн үндсэн классуудад дараах объектууд багтсан:

\begin{itemize}
  \item \textbf{Patient} — Өвчтөний нэр, нас, хаяг, утас зэрэг хувийн мэдээллийг хадгална.
  \item \textbf{Doctor} — Эмчийн нэр, хариуцсан тасгийн мэдээлэл.
  \item \textbf{Appointment} — \texttt{Patient} болон \texttt{Doctor} классуудыг холбон цаг захиалгыг илэрхийлнэ.
  \item \textbf{VisitRecord} — Онош, эмчилгээ, оношлогооны огноо зэрэг эмчилгээний түүхийг хадгална.
\end{itemize}

\begin{figure}[h]
    \centering
    \includegraphics[width=0.9\textwidth]{image (2).png}
    \caption{UML зураг}
    \label{fig:uml-diagram}
\end{figure}

\begin{figure}[h]
    \centering
    \includegraphics[width=0.4\textwidth]{506108377_1051333219852750_2026075398104957326_n.png}
    \caption{Кодын бүтэц}
    \label{fig:system-structure}
\end{figure}

\newpage

\section*{Patient класс}

\texttt{Patient(String name, int age, String phone)} \\
Шинэ өвчтөн бүртгэх зориулалттай конструктор. \\
\textbf{Параметрүүд:} нэр, нас, утасны дугаар. \\
\textbf{Алдаа:} Хэрэв нэр хоосон, нас 0-ээс бага, эсвэл утасны дугаар буруу бол \texttt{IllegalArgumentException}. \\
\textbf{Лог:} Амжилттай бүртгэгдсэн тухай \texttt{info} лог бүртгэгдэнэ.

\section*{Doctor класс}

\texttt{Doctor(String name, String department)} \\
Эмчийн нэр болон харьяалагдах тасгийн мэдээлэл бүхий обьект үүсгэнэ. \\
\textbf{Алдаа:} Нэр эсвэл тасгийн нэр хоосон байвал \texttt{IllegalArgumentException}. \\
\textbf{Лог:} Бүртгэлийн тухай \texttt{info} түвшний мэдээлэл хадгалагдана.

\section*{Appointment класс}

\texttt{Appointment(Patient patient, Doctor doctor, LocalDate date)} \\
Өвчтөн болон эмчийн хооронд цаг захиалга үүсгэнэ. \\
\textbf{Алдаа:}
\begin{itemize}
    \item Хоосон объект (\texttt{null}) дамжуулбал \texttt{NullPointerException}.
    \item Огноо өнгөрсөн бол \texttt{IllegalArgumentException}.
\end{itemize}
\textbf{Лог:} Амжилттай цаг захиалсан тухай \texttt{info} бүртгэгдэнэ.

\section*{VisitRecord класс}

\texttt{VisitRecord(Patient patient, String diagnosis, String treatment, LocalDate date)} \\
Өвчтөний эмчилгээний түүхийг бүртгэн хадгалах зориулалттай. \\
\textbf{Алдаа:} Өвчтөн \texttt{null} байвал эсвэл онош хоосон бол \texttt{IllegalArgumentException}. \\
\textbf{Лог:} Түүх бүртгэгдсэн тухай \texttt{info} лог хадгалагдана.

\section*{Log4j2 лог бүртгэлийн ашиглалт}

\texttt{org.apache.logging.log4j} санг ашиглан дараах лог түвшинг хэрэглэсэн:

\begin{itemize}
    \item \texttt{logger.info()} – Амжилттай үйлдлийн бүртгэл
    \item \texttt{logger.warn()} – Анхааруулга шаардлагатай нөхцөл
    \item \texttt{logger.error()} – Алдаа гарсан тохиолдолд
\end{itemize}

\section*{Алдааны зохицуулалт}

Систем дараах төрлийн алдааг илрүүлж, лог болон хэрэглэгчид ойлгомжтой мэдэгдэл үзүүлдэг:

\begin{itemize}
    \item \texttt{IllegalArgumentException} – Буруу утга, хоосон текст, нас 0-ээс бага гэх мэт
    \item \texttt{IllegalStateException} – Тухайн нөхцөлд тохироогүй үйлдэл хийх оролдлого
    \item \texttt{NullPointerException} – Хоосон объект дамжуулсан үед
\end{itemize}

\section*{Дүгнэлт}

Энэхүү \textbf{“Эмнэлгийн Өвчтөн Удирдлагын Систем”} төсөл нь эмнэлгийн өдөр тутмын үйл ажиллагааг програмчлалын аргаар үр дүнтэй удирдах боломжийг бүрдүүлсэн. Объект хандалтат зарчим, лог бүртгэл, алдааны зохицуулалт зэргийг ашигласнаар найдвартай, өргөтгөх боломжтой систем болсон.

Цаашид график хэрэглэгчийн интерфэйс нэмэх, эрхийн түвшин ялгах, мэдэгдэл илгээх зэрэг өргөтгөл хийх боломжтой.

\section*{Ашигласан материал}

\begin{itemize}
    \item \href{https://www.odoo.com}{Odoo.com} – Төслийн санаа
    \item \href{https://www.oracle.com/java/technologies/javase/jdk17-archive-downloads.html}{Java 17} – Програмчлалын хэл
    \item \href{https://maven.apache.org}{Maven} – Автоматжуулалтын хэрэгсэл
    \item \href{https://logging.apache.org/log4j/2.x/}{Log4j2} – Лог бүртгэлийн сан
    \item \href{https://junit.org/junit5/}{JUnit 5} – Тестийн фрэймворк
    \item \href{https://code.visualstudio.com}{VS Code} – IDE
    \item \href{https://github.com}{GitHub} – Код хадгалах, хуваалцах
    \item \href{https://www.microsoft.com/en-us/microsoft-teams/group-chat-software}{Microsoft Teams} – Даалгавар явуулах
\end{itemize}

\end{document}
